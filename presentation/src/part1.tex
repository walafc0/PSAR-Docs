  \section{Introduction}
    \bframe
      \begin{itemize}
        \item problématique:\newline \vspace{0.2cm}\hspace{0.4cm} architectures
          NUMA, placement mémoire, performances\newline
        \item<2> objectifs:\newline \vspace{0.2cm}\hspace{0.4cm} évaluation
          d'activité, mesures d'évènements, étude comportementale
        \end{itemize}
    \end{frame}

  \section{Architecture NUMA}
    \subsection{Présentation}

      \bframe
        \visible<1-> {
          \begin{block}{Objectifs}
            \begin{itemize}
              \item accélerer les temps de traitement
              \item répondre aux besoins d'applications spécifiques
            \end{itemize}
          \end{block}
        }
        \visible<2-> {
          \begin{block}{Moyens mis en \oe uvre}
            \begin{itemize}
              \item découpe en noeuds
              \item placement des contrôleurs d'E/S
              \item liens d'interconnexions
              \item mise en place d'une topologie
            \end{itemize}
          \end{block}
        }
      \end{frame}

    \subsection{Vue d'ensemble}
      \bframe
        \myfig{0.4}{img/numa_arch_details.png}
      \end{frame}

    \subsection{Enjeux}
      \bframe
        \begin{itemize}
          \item placement mémoire
          \item placement des threads
          \item activité d'entrées/sorties
        \end{itemize}
        \myfig{0.4}{img/topo.png}
      \end{frame}

  \section{Infrastructure de tests}
    \subsection{}
      \bframe
        \begin{itemize}
          \item utilisation mutualisée du Magny Cour $\rightarrow$ machines
            virtuelles
          \item problème: pas d'IBS avec qemu
        \end{itemize}
      
        \visible<2-> {
          \begin{alertblock}{Conséquence}
            Travail en réel sur le noyau pour 50\% du projet
          \end{alertblock}
        }
      \end{frame}



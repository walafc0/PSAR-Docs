\chapter{Mesures sur noyau Linux}

  \section{Introduction}

    Nous avons vu précédemment le fonctionnement des IBS. Dans cette partie, qui
    s'oriente plus vers l'aspect pratique du projet, nous allons expliquer la
    manière dont nous avons exploité les IBS. Comme les IBS peuvent renvoyer de
    très nombreuses informations, il nous faut réaliser une sélection des
    threads que nous allons analyser. Pour cela nous allons devoir mesurer leur
    activité et les trier de manière à sélectionner ceux qui nous intéressent le
    plus.

  \section{Définition de la chaleur d'un thread}

    Pour mesurer l'activité d'un thread nous avons décidé de lui associer un
    niveau de chaleur et un compteur qui lui est propre. C'est ce compteur que
    l'on va incrémenter ou décrémenter en fonction de son activité sur le
    système. Plus le thread est actif et plus son compteur augmente. Il est
    possible d'associer ce compteur de chaleur à différents événements en
    fonction des observations que l'on souhaite effectuer. Les différents
    évènements que nous avons jugé pertinents sont:

    \begin{itemize}
      \item le taux d'utilisation mémoire. Un processus utilisant beaucoup de
        mémoire est intéressant à analyser pour observer les stratégies de
        placement au niveaux des n\oe uds, et éventuellement changer cette
        stratégie.
      \item le nombre d'entrées/sortie. Si ce nombre est élevé, il est en
        général préférable de déplacer le thread responsable sur un des noeuds
        comportant un contrôleur d'E/S.
      \item les communications entre threads. Si deux threads éloignés ont
        besoin de s'échanger des informations, il est préférable de la
        rapprocher ou de les mettre sur le même n\oe pour diminuer le taux
        d'utilisation de la bande passante
    \end{itemize}


    \subsection{Méthodes de tri envisagées et inconvénients}

      Pour exploiter cette ressource nous avions dans un premier temps envisagé
      d'utiliser des structures dédiées dans lesquelles nous aurions stocké les
      informations utiles pour IBS. Cependant ces méthodes présentaient un
      certain nombre d'inconvénients.  La première structure imaginée était un
      tableau dont chaque case contenait l'identifiant du processus et la valeur
      de sa chaleur. Cela permettait un tri simple du tableau en fonction de la
      chaleur. Cependant les méthodes d'ajout de nouvelles cases dans un tableau
      borné ou de recherche des processus terminés selon des critères qui aurait
      été à déterminer n'était pas très optimales.  Pour simplifier l'ajout
      d'éléments nous avons alors pensé à l'utilisation de liste chaînée, mais
      qui n'apportait malheureusement aucune solution pour l'identification et
      le nettoyage des processus terminés. L'utilisation de liste chaînée
      rendait même moins performant la méthode de tri car le nombre d'éléments
      n'étant pas borné il pouvait croître et contenir des processus inactifs,
      inintéressant pour l'analyse mais qui aurait consommé des ressources lors
      du tri. Dans la seconde structure nous avions remplacé le compteur par un
      tableau où chaque case représentait un niveau de chaleur. Nous
      remplissions ensuite notre tableau avec une liste chaînée des identifiants
      des threads. Le tableau était ensuite actualisé, au lieu d'incrémenter un
      compteur on déplaçait un maillon d'une liste vers une liste plus haute du
      tableau ( et donc plus chaud). Pour le décrémenter on effectuait un
      décalage des listes chaînées vers le bas du tableau (plus froid). Ce type
      d'implémentation est peu flexible et ne présente aucune possibilité pour
      connaître la valeur du compteur à partir de l'identifiant du thread. Elle
      pose aussi des incertitudes concernant la pérennité du compteur car ce
      type de structure est indépendante.

    \subsection{Solution utilisée}

      Les structures envisagées ne permettant pas de gérer efficacement notre
      compteur, nous avons décidé pour remédier à ces inconvénients et ne pas
      avoir à redéfinir une structure de processus juste pour un compteur de
      modifier directement la structure task\_struct du noyau pour y ajouter
      notre compteur de chaleur. De cette manière notre compteur est directement
      rattaché à son processus et ne peut être détruite ou modifiée par erreurs.


  \section{Tri des threads en fonction de leur chaleur}

    En abandonnant l'utilisation d'un tableau ou d'une liste pour stocker la
    chaleur d'un thread, nous devons trouver une méthode qui permettra de
    classer les processus en fonction de leur chaleur contenu dans la
    task\_struct.

    \subsection{Structure du système de tri}

      De manière à maîtriser la valeur du compteur nous lui avons fixé des
      bornes. Il devient alors possible de classer les threads dans un tableau
      ou chaque case sera associée à une chaleur et pointera vers une liste
      contenant tous les threads de même chaleur comme pour notre deuxième
      structure envisagées.  La liste est simplement chaînée et se compose de
      structures contenant un pointeur vers la task\_struct représentée et un
      pointeur vers l'élément suivant. Nous avions envisagé d'ajouter un
      pointeur directement dans la task\_struct cependant cela obligeait à gérer
      la cohérence de la liste si la task\_struct est supprimée entre temps.
      L'implémentation de ce tableau de chaleurs consiste à parcourir la
      task\_struct et à tester si l'état du processus correspond aux éventements
      que nous voulons observer. Chaque processus commence avec une chaleur
      nulle et si son état correspond à notre événement on incrémente de manière
      fixe la valeur de sa chaleur, sinon on décrémente son niveau de chaleur,
      puis on place le thread dans la table de chaleur.  Cette table de chaleur
      est généré à intervalle régulier de manière à classer l'ensemble des
      task\_struct. Cependant seul les éléments les plus actif nous intéressent,
      nous allons donc créer un autre tableau de taille correspondant au nombre
      de threads à monitorer dans lequel nous stockerons le pid et des
      informations nécessaires aux mesures avec IBS des threads identifiés comme
      les plus chaud dans notre table de chaleur. Une fois mis de coté il est
      possible de vider le tableau de liste pour la prochaine utilisation.

    \subsection{Evolution du compteur de chaleur}

      Une fois la structure mise en place il est nécessaire de s’intéresser à
      l'évolution du compteur qui déterminera le classement des threads.  Dans
      notre cas nous avons décidé, comme critère de chaleur, de tester si l'état
      du processus correspond à \og running\fg, c'est à dire de tester si il est
      en cours d’exécution sur le processeur. Si c'est le cas on incrémente la
      valeur de notre compteur. Cela permet de définir comme chaud des processus
      ayant une forte occupation du cpu.  Pour optimiser ce fonctionnement et
      affiner la liste des threads chauds dans notre classement il est possible
      de gérer dynamiquement la valeur dont le compteur va être décrémenté. En
      effet, si le système produit une grande quantité de threads très actif une
      gestion statique de notre compteur risque d'amasser peu à peu les
      processus sur la borne haute de notre table de chaleur chaleur. Nous
      aurions alors une grande quantité de threads chaud à un moment donné mais
      qui sur la durée ne se révéleraient pas forcément très utile pour une
      analyse plus poussée avec IBS qui fourniraient des données peu concluantes
      ou erronées.  Or un thread qui nous intéresse n'est pas un thread avec une
      chaleur importante mais avec une chaleur plus importante que celle des
      autres threads. Il est donc intéressant de pouvoir contrôler et affiner à
      la volée la quantité de threads se trouvant parmi les éléments chaud de
      notre table. La décrémentation dynamique permet de remplir ce rôle en
      comptant par exemple le nombre de threads référencés dans la ligne la plus
      chaude de notre table. On peut choisir comme critère que l'on n'a pas
      besoin d'avoir plus du double en éléments que ce que l'on peut sauvegarder
      dans notre tableau de résultat et ensuite accélerer progressivement la
      vitesse de décrémentation ou au contraire lorsqu'il n'y en a plus assez
      pour remplir le tableau réduire la vitesse de décrémentation pour
      permettre au processus de revenir dans le haut de la table.  Nous venons
      de voir la méthode que nous avons élaborer pour sélectionner les processus
      ayant une plus forte probabilité de fournir des résultats satisfaisant et
      ayant un intérêt pour ensuite être observé avec IBS. Dans la suite de ce
      rapport nous allons présenter le fonctionnement de notre programme, la
      manière dont les mesures sont lancés et les problèmes que nous avons
      rencontrés.


  \section{Réalisation des mesures}

    Nous avons à présent tous les éléments nécessaire à la création de notre
    programme. Dans cette partie nous allons présenter le fonctionnement de
    notre fonction principal chargé d'initialiser les mesures et de lancer les
    IBS.

    \subsection{Lancement des mesures : thread\_fn()}

      La fonction thread\_fn() est la fonction qui nous permet d'exploité la
      table de chaleur vu, d'initialiser les différents paramètres et de lancer
      les mesures.  Cette fonction touchant à des éléments du noyau il est
      nécessaire de poser quelques sécurités comme par exemple l'activation des
      signaux, de définir le thread exécutant ce code comme interruptibles, et
      de fixer des timeouts. Cela permet en cas de problèmes de laisser le noyau
      se rétablir et tenter de continuer son fonctionnement normal.  La seconde
      phase consiste à initialiser différentes méthodes comme les IBS.
      Seulement une fois le système initialiser il nous est possible de
      commencer les mesures.  Comme nous l'avons précisé précédemment notre
      critère de chaleur se base sur l'occupation par un thread du
      processeur. Nous commençons donc par parcourir la table de tous les
      processus pour trouver ceux dont l'état est à running, s'ils le sont on
      incrémente leur chaleur, sinon elle décroit mais la chaleur ne peut pas
      être inférieur à 0.  Nous effectuons ce parcours de table en boucle
      pendant un certain temps. Ce temps est défini par une variable que l'on
      incrémente à chaque parcours de la table et qui une fois une certaine
      valeur atteinte nous permet de sortir de la boucle. L’intérêt de cette
      boucle est de remplir la table de chaleur et de faire évoluer son contenu
      pendant que d'autre mesures comme les IBS peuvent être exécutées.  L'étape
      suivante permet de préparer le lancement de nouvelles mesures IBS. Si des
      mesures étaient déjà en cours, on les arrête toutes et on purge notre
      table de chaleur pour lui faire accueillir les nouvelles valeurs de
      chaleurs obtenu lors du parcours précédent de la table des processus.  Il
      faut à présent remplir notre table de chaleur pour cela nous allons
      parcourir toutes les task\_struct et remplir notre table en classant
      chaque processus à l'aide des valeurs de chaleurs que l'on trouve dans
      leur task\_struct. On enchaîne directement avec la recherche des threads
      les plus chaud dans notre table pour les sauvegarder immédiatement dans
      notre tableau dédié aux résultats en y indiquant le pid des threads chaud
      ainsi que le numéro du cpu à observer avec IBS pour obtenir nos résultats.
      Nous allons à présent parcourir ce tableau de résultat et lancer pour
      chaque thread les mesures IBS.  La première exécution de la fonction
      principale est terminé, nous exécutons ce déroulement en boucle pour
      obtenir nos mesures
	\section{Mesures}
		Voici ci-dessous un exemple des mesures récoltées dans la fonction de traitement d'une interruption IBS. Les valeurs sont affichées par des printk, ce sont des entiers sur 32 bits c'est à dire qu'elles ne donnent pas directement les informations concernant les bits activés dans les registres de données. Pour les obtenir il faut simplement convertir en binaire les entiers du résultat et voir quels sont les bits activés lors de la mesure.
		\begin{verbatim}

amd48-systeme kernel: [ 2579.394384] MSR_AMD64_IBSOPDATA2 0 0
amd48-systeme kernel: [ 2579.394391] MSR_AMD64_IBSOPRIP -2130035521 -1
amd48-systeme kernel: [ 2579.394393] MSR_AMD64_IBSOPDATA 1179659 0
amd48-systeme kernel: [ 2579.394395] MSR_AMD64_IBSOPDATA3 0 0
amd48-systeme kernel: [ 2579.394396] MSR_AMD64_IBSOPCLINAD 0 0
amd48-systeme kernel: [ 2579.394397] MSR_AMD64_IBSDCPHYSAD 0 0
amd48-systeme kernel: [ 2579.394399] MSR_AMD64_IBSOPCTL 925695 0

amd48-systeme kernel: [ 2579.424173] MSR_AMD64_IBSOPDATA2 0 0
amd48-systeme kernel: [ 2579.424180] MSR_AMD64_IBSOPRIP -2130432956 -1
amd48-systeme kernel: [ 2579.424182] MSR_AMD64_IBSOPDATA 655364 0
amd48-systeme kernel: [ 2579.424183] MSR_AMD64_IBSOPDATA3 394273 0
amd48-systeme kernel: [ 2579.424185] MSR_AMD64_IBSOPCLINAD 1073729000 -30719
amd48-systeme kernel: [ 2579.424187] MSR_AMD64_IBSDCPHYSAD 1073729000 1
amd48-systeme kernel: [ 2579.424188] MSR_AMD64_IBSOPCTL 925695 0

amd48-systeme kernel: [ 2579.424277] MSR_AMD64_IBSOPDATA2 0 0
amd48-systeme kernel: [ 2579.424279] MSR_AMD64_IBSOPRIP -2130593607 -1
amd48-systeme kernel: [ 2579.424281] MSR_AMD64_IBSOPDATA 1441814 0
amd48-systeme kernel: [ 2579.424282] MSR_AMD64_IBSOPDATA3 0 0
amd48-systeme kernel: [ 2579.424283] MSR_AMD64_IBSOPCLINAD 0 0
amd48-systeme kernel: [ 2579.424284] MSR_AMD64_IBSDCPHYSAD 0 0
amd48-systeme kernel: [ 2579.424286] MSR_AMD64_IBSOPCTL 925695 0

amd48-systeme kernel: [ 2579.424345] MSR_AMD64_IBSOPDATA2 0 0
amd48-systeme kernel: [ 2579.424347] MSR_AMD64_IBSOPRIP -2127328833 -1
amd48-systeme kernel: [ 2579.424348] MSR_AMD64_IBSOPDATA 1179664 0
amd48-systeme kernel: [ 2579.424350] MSR_AMD64_IBSOPDATA3 0 0
amd48-systeme kernel: [ 2579.424351] MSR_AMD64_IBSOPCLINAD 0 0
amd48-systeme kernel: [ 2579.424352] MSR_AMD64_IBSDCPHYSAD 0 0
amd48-systeme kernel: [ 2579.424353] MSR_AMD64_IBSOPCTL 925695 0

amd48-systeme kernel: [ 2579.434233] MSR_AMD64_IBSOPDATA2 0 0
amd48-systeme kernel: [ 2579.434238] MSR_AMD64_IBSOPRIP -2130111010 -1
amd48-systeme kernel: [ 2579.434239] MSR_AMD64_IBSOPDATA 1114125 0
amd48-systeme kernel: [ 2579.434241] MSR_AMD64_IBSOPDATA3 393250 0
amd48-systeme kernel: [ 2579.434242] MSR_AMD64_IBSOPCLINAD 936245760 -30717
amd48-systeme kernel: [ 2579.434244] MSR_AMD64_IBSDCPHYSAD 936245760 3
amd48-systeme kernel: [ 2579.434245] MSR_AMD64_IBSOPCTL 925695 0

amd48-systeme kernel: [ 2579.434557] MSR_AMD64_IBSOPDATA2 0 0
amd48-systeme kernel: [ 2579.434562] MSR_AMD64_IBSOPRIP -2130435850 -1
amd48-systeme kernel: [ 2579.434565] MSR_AMD64_IBSOPDATA 16842754 0
amd48-systeme kernel: [ 2579.434566] MSR_AMD64_IBSOPDATA3 278528 0
amd48-systeme kernel: [ 2579.434567] MSR_AMD64_IBSOPCLINAD 0 0
amd48-systeme kernel: [ 2579.434569] MSR_AMD64_IBSDCPHYSAD 268483584 0
amd48-systeme kernel: [ 2579.434570] MSR_AMD64_IBSOPCTL 925695 0
		\end{verbatim}

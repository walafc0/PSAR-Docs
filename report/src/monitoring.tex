\chapter{Monitoring}
	\section{Introduction}
		Le monitoring est le fait de suivre l'éxécution d'un programme pour savoir quelles sont les ressources utilisées par une certaine entité. Le monitoring permet de connaître les ressources qu'utilise un système de sorte à ce que l'éxécution d'un programme puisse être optimisée et se dérouler de manière plus performante. Il existe deux manières principales de faire du monitoring : 
		\benum
			\item{Choisir un type d'évènements à compter, chaque évènement qui occure va incrémenter un compteur d'évènement qui sera ensuité récolté}
			\item{Executer le suivi d'un instruction précise et repertorier les effets que cette instruction incombe sur le système}
		\eenum
		Grâce au monitoring nous pouvons connaître les effets précis d'un programme sur le systeme, les évènement généralement intéréssants à surveiller sont les suivants :
		\benum
			\item{Fautes de cache}
			\item{Etat de branchement d'une instruction}
			\item{Entrées/Sorties provoquées par une instruction}
		\eenum
		Les processeurs récents mettent à disposition des outils qui permettent d'effectuer des mesures sur les instructions qu'ils effectuent. Ces outils sont des programmes miniatures, implantés en dur sur le processeur, et utilisent pour leur éxécution des registres dédiés à cette utilisation. Par conséquent, nous allons devoir agir sur ces registres dédiés, certains dans lesquels nous allons lire pour récupérer les informations qui nous seront nécéssaires, d'autres dont il faudra modifier les valeurs pour mettre en place les mesures. Les registres utilisés ici sont de type Model Spécific Register (MSR), ils vont donc permettre l'interaction entre le système d'exploitation et le materiel. Les MSR sont donc spécifiques à chaque architecture et peuvent être liés à des fonctionnalités de debuggage, de monitoring, mais le système peut également s'en servir pour activer des fonctionnalités du processeur, donner des informations de temps (cycles CPU, timestamp counter), et peuvent être utilisés pour de la gestion d'erreur. AMD sur ses processeurs fournit 2 types de MSR : 
		\benum
			\item{MSR Legacy : un pannel de registres communs à tous les modèles de processeurs, permet de définir une norme}
			\item{Les autres MSR : qui sont eux, spécifiques à chaque modèle de processeur}
		\eenum
	\section{Performance Monitoring Counters}
		Le procédé Performance Monitoring Counters utilise des MSR Legacy et est donc implanté sur la plupart des modèles de processeurs. C'est un type de monitoring où l'on mesure la fréquence d'apparition de certains évènements prédéfinis de base dans la documentation du processeur. Lorsqu'un évènement mesuré se produit, le compteur qui lui est associé est incrémenté. Par conséquent, régulièrement, le processeur lève une interruption et une fonction définie au préalable va s'occuper de venir récupérer les valeurs enregisrées dans ces registres. D'autre part, les Performance Monitoring Counter engendrent certains problèmes de cohérence lorsqu'ils sont mis en ouvre sur des processeurs multi-coeur, décrit plus bas. (cf. : Challenges du monitoring)
	\paragraph{Implémentation}
		L'architecture Legacy pour la mise en place des \PMC met à disposition 4 MSR pour le contrôle des evenements et respectivement le même nombre de registres pour le résultat. Il existe pour certains types des processeurs des extensions permettant des mesures plus poussées, en rapport avec le NorthBridge. Chaque MSR contrôle un évènement distinct et lorsque l'évènement se produit, le registre associé est incrémenté. Le traitement à effectuer dans la fonction de traitement (handler) est donc de récupérer l'information stockée dans le MSR voulu et le remettre à 0 afin de ne pas fausser les mesures suivantes.
	\section{Instruction Based Sampling}

\pagestyle{plain}
\addcontentsline{toc}{chapter}{Conclusion}
\chapter*{Conclusion}

  Pour conclure ce projet, nous voulons avant tout revenir sur les connaissances
  acquises lors de ce dernier. Les architectures NUMA et toutes les
  problématiques liées nous étaient totalement inconnues il y a encore 5
  mois. Leur découverte fut tout aussi intéressante qu'enrichissante. Nous avons
  pu ainsi, à travers la lecture de nombreuses documentations, appréhender un
  nouveau sujet de recherche prometteur mais également prendre conscience de la
  complexité d'un ordinateur. Si celui-ci est réduit à un boitier pour la
  plupart des gens, nous nous sommes rendus compte qu'il existe des
  problématiques bien plus profondes que cela, et notamment matérielles. Avant
  même la conception d'un noyau, il est possible de réfléchir à l'amélioration
  de ses performances rien qu'en utilisant une autre architecture.\newline


  De plus, grâce aux cours d'architecture des ordinateurs de Mr. Bazargan et
  ceux de noyau de Mr. Sens, nous avons pu mettre en applications nos
  compétences pour réaliser un module pour le noyau Linux permettant la mesure
  d'évènements avec la bibliothèque IBS d'AMD. Bien que ce module ne soit pas
  terminé, nous sommes satisfaits du résultat et nous espérons l'améliorer
  grandement lors de nos stages cet été au laboratoire. Ce projet nous aura
  également permis de confirmer nos envies respectives d'étudier les systèmes
  d'exploitation, de comprendre leur fonctionnement, les différents types de
  systèmes, de noyaux et d'architectures.\newline

  Enfin, on peut se poser la question si, à l'heure actuelle, aux vues de
  l'augmentation de la rapidité des processeurs, ne serait-il pas plus judicieux
  de concentrer les efforts sur la création de mémoires beaucoup plus rapides
  plutôt que de vouloir sans cesse augmenter le nombres de coeur au sein d'une
  machine ?





  \section{Mesures sur le noyau Linux}
    \subsection{Chaleur d'un thread}
      \bframe
        \begin{itemize}
          \item un compteur représente l'activité d'un thread
          \item différents critères d'activité:
            \begin{itemize}
              \item état: (in)actif
              \item taux d'utilisation mémoire
              \item nombre d'entrées/sorties
              \item commnunications entre threads
              \item \ldots
            \end{itemize}
        \end{itemize}
      \end{frame}

    \subsection{Méthodes de tri envisagées}
      \bframe
        \begin{itemize}
          \item nécessité d'une structure dédiée
          \item utilisation d'un tableau ou d'une liste chainée
          \begin{itemize}
            \item insertion de nouveaux threads
            \item difficulté à trouver les threads morts
            \item tri peu performant
          \end{itemize}
        \end{itemize}
        \visible<2-> {
          \begin{alertblock}{Conclusion}
            Solution abandonnée
          \end{alertblock}
        }
      \end{frame}

      \bframe
          Utilisation d'un tableau de chaleur
          \myfig{0.25}{img/screen1.png}
      \end{frame}
      \bframe
          Utilisation d'un tableau de chaleur
          \myfig{0.25}{img/screen2.png}
      \end{frame}
      \bframe
          Utilisation d'un tableau de chaleur
          \myfig{0.25}{img/screen3.png}
      \end{frame}
      \bframe
          Utilisation d'un tableau de chaleur
          \myfig{0.25}{img/screen4.png}
      \end{frame}

    \subsection{Solution retenue}
      \bframe
        \begin{itemize}
          \item ajout du compteur dans la task\_struct
          \item on conserve le tableau de chaleur précédent
          \item structure Gestion pour les listes
        \end{itemize}
        \myfig{0.3}{img/screen5.png}
      \end{frame}

      \bframe
        \myfig{0.33}{img/screen6.png}
      \end{frame}

    \subsection{Réalisation}

      \bframe
        Algorithme:
        \begin{itemize}
          
          \item[1] parcourir tous les threads
          \begin{itemize}
            \item[a] si RUNNING $\rightarrow$ incrémentation du compteur de chaleur
            \item[b] sinon décrémentation
          \end{itemize}
          \item[2] stopper IBS
          \item[3] vider le tableau de chaleurs
          \item[4] générer le tableau de chaleurs
          \item[5] lancer les mesures sur les threads chauds
        \end{itemize}
      \end{frame}

      \bframe
        \begin{block}{Optimisation}
          \begin{itemize}
            \item Utilisation d'un facteur d'incrémentation et de décrémentation
              dynamique
          \end{itemize}
        \end{block}
        \visible<2-> {
          \begin{alertblock}{Problèmes}
            \begin{itemize}
            \item pas d'IBS avec qemu $\rightarrow$ merge impossible sur Magny Cour
            \end{itemize}
          \end{alertblock}
        }
      \end{frame}

  \section{Conclusion}
    \subsection{}
      \bframe
        \begin{block}{Apports personnels}
          \begin{itemize}
            \item beaucoup de connaissances acquises
            \item utile pour l'année prochaine
            \item découverte d'une nouvelle architecture prometteuse
          \end{itemize}
        \end{block}
        \begin{block}{Ce qu'il reste à faire}<2>
          \begin{itemize}
            \item merger les deux parties du projet sur Magny Cour
            \item mettre en place un traitement des données
            \item améliorer l'algorithme de tri d'activités
          \end{itemize}
        \end{block}
      \end{frame}

   \section{The end}
	\bframe
		\begin{center}To be continued...\\
		No questions please.\end{center}
                \myfig{0.003}{img/jackiechan.jpg}
	\end{frame}


      %% \bframe
      %%   20/20 ou chinois tuer et manger ta famille !
      %%   \begin{figure}[h]
      %%     \centering
      %%     \includegraphics[scale=0.4]{img/jackiechan.jpg}\newline
      %%     \textit{``Tu veux un bol de riz ?''}
      %%   \end{figure}
      %% \end{frame}



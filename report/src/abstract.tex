L’essor de l’informatique en nuage a permis aux administrations et entreprises
de stocker d’énormes jeux de données. Aujourd’hui, l’un des goulots
d’étranglement majeurs pour les performances de traitement de ces données est le
système d’exploitation de chaque machine. Les systèmes actuels ne peuvent pas
gérer efficacement les applications intensives en données car ils ne disposent
pas d’une vue unifiée des ressources utilisées, ce qui les empêche de déterminer
des stratégies efficaces pour le placement des tâches/données sur les ressources
matérielles. Une meilleure gestion des ressources permettrait une forte
réduction du nombre de machines nécessaires aux traitements des données.

L’implémentation de sondes dans le noyau Linux permettrait d'identifier les
ressources physiques et logicielles les plus sollicitées par les processus. Les
informations remontées par ces sondes peuvent ensuite permettre de commencer à
définir des stratégies de placement des tâches et des données prenant en compte
à la fois la topologie de la machine et l’utilisation effectives des ressources
par les tâches.
